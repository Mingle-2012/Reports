\documentclass{article}
\usepackage{fancyhdr}
\usepackage{ctex}
\usepackage{listings}
\usepackage[a4paper, body={18cm,22cm}]{geometry}
\usepackage{amsmath,amssymb,amstext,wasysym,enumerate,graphicx}
\usepackage{float,abstract,booktabs,indentfirst,amsmath}
\usepackage{multirow}
\usepackage{enumitem}
\usepackage{listings}
\usepackage{xcolor}
\usepackage{tabularx}
\usepackage[most, skins, breakable, theorems]{tcolorbox}
\usepackage{accsupp}
\usepackage[bottom]{footmisc}
\usepackage{subcaption}
\usepackage{logicproof}
% \usepackage[backend=biber,style=numeric]{biblatex}
\usepackage[xetex]{hyperref}
\usepackage{fontspec}
\usetikzlibrary{arrows.meta}
\newcommand\emptyaccsupp[1]{\BeginAccSupp{ActualText={}}#1\EndAccSupp{}}
\setlength{\parindent}{2em}
\setmonofont{Consolas}
\setCJKmonofont{黑体}
% \setmainfont{Times New Roman}
\hypersetup{CJKbookmarks=true,colorlinks=true,citecolor=blue,%
            linkcolor=blue,urlcolor=blue,bookmarksnumbered=true,%
            bookmarksopen=true,breaklinks=true}
\lstset{
    % language = C,
    xleftmargin = 3em,xrightmargin = 3em, aboveskip = 1em,
	backgroundcolor = \color{white}, % 背景色
	basicstyle = \small\ttfamily, % 基本样式 + 小号字体
	rulesepcolor= \color{gray}, % 代码块边框颜色
	breaklines = true, % 代码过长则换行
	numbers = left, % 行号在左侧显示
	numberstyle = \small\emptyaccsupp, % 行号字体
    numbersep = -14pt, 
    keywordstyle=\color{purple}\bfseries, % 关键字颜色
    commentstyle =\color{red!50!green!50!blue!60}, % 注释颜色
    stringstyle = \color{red}, % 字符串颜色
    morekeywords={ASSERT, int64_t, uint32_t},
	frame = single, 
	showspaces = false, % 不显示空格
    showstringspaces = false,
	columns = fixed, % 字间距固定
    literate=
        {^-}{{{\color{black}\textbf{\color{red}-}}\colorbox{red!30}{\phantom{XX}}}}{1}
        {^+}{{{\color{black}\textbf{\color{green}+}}\colorbox{green!30}{\phantom{XX}}}}{1},
}

\raggedbottom

\title{\heiti\textbf{计算机逻辑基础}}
\author{第四次作业 \\ 
武泽恺 10225101429
}
\date{2025年11月8日}

\begin{document}
\maketitle


\section*{Theorem 1}
$$
\forall x(\exists y\,P(x,y)\to Q(x))\ \vdash\ \forall x(\forall y\,P(x,y)\to Q(x))
$$

\begin{logicproof}{3}
  \forall x(\exists y P(x, y) \to Q(x)) & Premise \\
  \begin{subproof}
    x_0 & Assumption \\
    \exists y P(x_0, y) \to Q(x_0) & Assumption \\
    \begin{subproof}
        y_0 & Assumption \\
        P(x_0, y_0) \to Q(x_0) & Assumption \\
        \begin{subproof}
            \forall y P(x_0, y) & Assumption \\
            P(x_0, y_0) & $\forall$e (6) \\
            Q(x_0) & $\to$e (5,7)
        \end{subproof}
        \forall y P(x_0, y) \to Q(x_0) & $\to$i (6--8)
    \end{subproof}
    \forall y P(x_0, y) \to Q(x_0) & $\exists$e (3,9)
  \end{subproof}
  \forall x(\forall y P(x, y) \to Q(x)) & $\forall$i (2--10)
\end{logicproof}


\section*{Theorem 2}
$$
\forall x(P(x)\leftrightarrow x=b)\ \vdash\ P(b)\wedge \forall x\forall y((P(x)\wedge P(y))\to x=y).
$$

\begin{logicproof}{3}
  \forall x(P(x) \to x = b) & Premise \\
  \forall x(x = b \to P(x)) & Premise \\
  b = b \to P(b) & $\forall$ e (2) \\
  b = b & = i \\ % 4
  P(b) & $\to$ e (4, 3) \\ % 5
  \begin{subproof}
    x_0 & Assumption \\ % 6 (Variable Introduction)
    y_0 & Assumption \\ % 6 (Variable Introduction)
    \begin{subproof}
      P(x_0) \land P(y_0) & Assumption \\ % 7
      P(x_0) & $\land$ $e_1$ (8) \\ % 8
      P(x_0) \to x_0 = b & $\forall$ e (1) \\ % 9
      x_0 = b & $\to$ e (9, 10) \\ % 10
      \begin{subproof}
        P(y_0) & $\land$ $e_2$ (8) \\ % (Implicit step for line 11)
        P(y_0) \to y_0 = b & $\forall$ e (1) \\
        y_0 = b & $\to$ e (12, 13)
      \end{subproof}
      x_0 = y_0 & = e (12, 14)
    \end{subproof}
    P(x_0) \land P(y_0) \to x_0 = y_0 & $\to$ i (8 - 15)
  \end{subproof}
  \forall x \forall y(P(x) \land P(y) \to x = y) & i (6 - 16) \\
  P(b) \land \forall x \forall y(P(x) \land P(y) \to x = y) & $\land$ i (5, 17)
  
\end{logicproof}

\end{document}