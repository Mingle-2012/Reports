\documentclass{article}
\usepackage{fancyhdr}
\usepackage{ctex}
\usepackage{listings}
\usepackage[a4paper, body={18cm,22cm}]{geometry}
\usepackage{amsmath,amssymb,amstext,wasysym,enumerate,graphicx}
\usepackage{float,abstract,booktabs,indentfirst,amsmath}
\usepackage{multirow}
\usepackage{enumitem}
\usepackage{listings}
\usepackage{xcolor}
\usepackage{tabularx}
\usepackage[most, skins, breakable, theorems]{tcolorbox}
\usepackage{accsupp}
\usepackage[bottom]{footmisc}
\usepackage{subcaption}
\usepackage{logicproof}
% \usepackage[backend=biber,style=numeric]{biblatex}
\usepackage[xetex]{hyperref}
\usepackage{fontspec}
\usepackage{amsthm}
\newtheorem{theorem}{Theorem}
\usetikzlibrary{arrows.meta}
\newcommand\emptyaccsupp[1]{\BeginAccSupp{ActualText={}}#1\EndAccSupp{}}
\setlength{\parindent}{2em}
\setmonofont{Consolas}
\setCJKmonofont{黑体}
% \setmainfont{Times New Roman}
\hypersetup{CJKbookmarks=true,colorlinks=true,citecolor=blue,%
            linkcolor=blue,urlcolor=blue,bookmarksnumbered=true,%
            bookmarksopen=true,breaklinks=true}
\lstset{
    % language = C,
    xleftmargin = 3em,xrightmargin = 3em, aboveskip = 1em,
	backgroundcolor = \color{white}, % 背景色
	basicstyle = \small\ttfamily, % 基本样式 + 小号字体
	rulesepcolor= \color{gray}, % 代码块边框颜色
	breaklines = true, % 代码过长则换行
	numbers = left, % 行号在左侧显示
	numberstyle = \small\emptyaccsupp, % 行号字体
    numbersep = -14pt, 
    keywordstyle=\color{purple}\bfseries, % 关键字颜色
    commentstyle =\color{red!50!green!50!blue!60}, % 注释颜色
    stringstyle = \color{red}, % 字符串颜色
    morekeywords={ASSERT, int64_t, uint32_t},
	frame = single, 
	showspaces = false, % 不显示空格
    showstringspaces = false,
	columns = fixed, % 字间距固定
    literate=
        {^-}{{{\color{black}\textbf{\color{red}-}}\colorbox{red!30}{\phantom{XX}}}}{1}
        {^+}{{{\color{black}\textbf{\color{green}+}}\colorbox{green!30}{\phantom{XX}}}}{1},
}

\raggedbottom

\title{\heiti\textbf{计算机逻辑基础}}
\author{第五次作业 \\ 
武泽恺 10225101429
}
\date{2025年11月16日}

\begin{document}
\maketitle


\section*{Assignment 5-1}
\subsection*{Problem} 
Read the following specification: Alice, Bob, and Charles live in an apartment. One day Alice is found dead in her room. Please translate the following clues into predicate logic formulas:

\begin{enumerate}
    \item A killer always hates, and is no richer than, his victim.
    \item Charles hates no one that Alice hates.
    \item Alice hates everybody except Bob.
    \item Bob hates everyone who is either not richer than Alice or hated by Alice.
    \item No one hates everyone.
    \item No one is richer than himself or herself. Who is richer can be determined for any two different persons.
\end{enumerate}

Guess who is the killer and verify your conclusion with semantical entailment.

\subsection*{Answer}

我们可以用$\{x, y, z\}$表示任意人,用常量符号 $\{a, b, c\}$ 分别表示 Alice、Bob 和 Charles。
\[
K(x,y): x \text{ killed } y, \qquad 
H(x,y): x \text{ hates } y, \qquad 
R(x,y): x \text{ is richer than } y.
\]

我们可以将题目中的线索翻译为以下谓词逻辑公式:
\begin{align}
(1)\quad &\forall x \forall y \bigl(K(x,y) \rightarrow (H(x,y) \land \neg R(x,y))\bigr). \\[4pt]
(2)\quad &\forall z\bigl(H(a,z) \rightarrow \neg H(c,z)\bigr). \\[4pt]
(3)\quad &\forall z\bigl(H(a,z) \leftrightarrow z \neq b\bigr). \\[4pt]
(4)\quad &\forall z\bigl((\neg R(z,a)\lor H(a,z)) \rightarrow H(b,z)\bigr). \\[4pt]
(5)\quad &\forall x\, \exists y\, \neg H(x,y). \\[4pt]
(6)\quad &\forall x\, \neg R(x,x), 
\qquad 
\forall x\forall y\,(x\neq y \rightarrow (R(x,y)\lor R(y,x))). \\[4pt]
&\exists x\,K(x,a) \qquad\text{(Alice 被杀)}
\end{align}

我们可以推测,Alice 是唯一可能杀死她自己的人。

\begin{proof}
从公理 (3) 中令 $z=a$ 得到 $H(a,a)$(Alice 恨 Alice)。
接着,公理 (2) 蕴含 $\neg H(c,a)$;因此 Charles 不恨 Alice。
根据公理 (1),任何杀人者必须恨其受害者,所以 Charles 不可能是杀人者。

假设 $K(x,a)$ 对某个 $x$ 成立(即某人杀了 Alice)。
则由公理 (1) 可知,我们得到 $H(x,a)$ 且 $\neg R(x,a)$。

考虑 $x=b$ (Bob)。假设 $\neg R(b,a)$ 成立,那么由公理 (4) 可知,对于任意 $z$,因为对于所有 $z\neq b$,公理 (3) 保证 $H(a,z)$ 成立;而对于 $z=b$,$\neg R(b,a)$ 成立。因此,前提 $(\neg R(z,a)\lor H(a,z))$ 都为真,公理 (4) 蕴含 $H(b,z)$ 对所有 $z$ 都成立(Bob 恨所有人)。
这与公理 (5) 矛盾,公理 (5) 表明没有人恨所有人。
因此,Bob 不可能满足 $\neg R(b,a)$,从而不能是杀人者。

因此,Bob 和 Charles 都不能是杀人者;剩下的唯一嫌疑人是 Alice 自己。
\end{proof}


\section*{Assignment 5-2}

\subsection*{(a)}
$$
\neg \forall x. P(x) \equiv \exists x. \neg P(x)
$$

我们的目标是证明对任意模型 $M$,都有
$$
\forall M. (M \models \neg \forall x. P(x) \iff M \models \exists x. \neg P(x))
$$

\begin{proof}
    \begin{align*}
M \models \neg \forall x. P(x)
&\equiv_{df} \neg \forall a. M, l[x \mapsto a] \models P(x) \iff \exists a. M, l[x \mapsto a] \models \neg P(x) \\
&\equiv_{df} M \models \exists x. \neg P(x)
\end{align*}
\end{proof}

\subsection*{(b)}
$$
\exists x \forall y. P(x,y) \not\equiv \forall y \exists x. P(x,y)
$$

我们的目标是找到一个模型 $M$,使得
\[
\exists M. M \models \forall y \exists x. P(x, y) \And M \not\models \exists x \forall y. P(x, y)
\]

\begin{proof}

我们选取论域 $D = \{0, 1\}$,并定义谓词 $P$ 的解释 $P^M = \{(0, 0), (1, 1)\}$,即$P(0, 0)$ 和 $P(1, 1)$ 为真;$P(0, 1)$ 和 $P(1, 0)$ 为假。

1. 验证 $\forall y \exists x. P(x, y)$ 为真:
\begin{itemize}
    \item 当 $y=0$ 时,我们选择 $x=0$,则 $P(0, 0)$ 成立。
    \item 当 $y=1$ 时,我们选择 $x=1$,则 $P(1, 1)$ 成立。
\end{itemize}
由于论域中所有的 $y$ 都找到了对应的 $x$ 使得关系成立,因此 $M \models \forall y \exists x. P(x, y)$ 为真。

2. 验证 $\exists x \forall y. P(x, y)$ 为假:
\begin{itemize}
    \item 检查 $x=0$: $P(0, 0)$ 成立,但 $P(0, 1)$ 在 $M$ 中为假。因此,$x=0$ 失败。
    \item 检查 $x=1$: $P(1, 1)$ 成立,但 $P(1, 0)$ 在 $M$ 中为假。因此,$x=1$ 失败。
\end{itemize}
由于 $D$ 中不存在一个 $x$ 能使得 $P(x, y)$ 对所有 $y$ 都成立,因此 $M \not\models \exists x \forall y. P(x, y)$ 为假。
\end{proof}


\end{document}