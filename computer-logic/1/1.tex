\documentclass{article}
\usepackage{fancyhdr}
\usepackage{ctex}
\usepackage{listings}
\usepackage[a4paper, body={18cm,22cm}]{geometry}
\usepackage{amsmath,amssymb,amstext,wasysym,enumerate,graphicx}
\usepackage{float,abstract,booktabs,indentfirst,amsmath}
\usepackage{multirow}
\usepackage{enumitem}
\usepackage{listings}
\usepackage{xcolor}
\usepackage{tabularx}
\usepackage[most]{tcolorbox}
\usepackage{accsupp}
\usepackage[bottom]{footmisc}
\usepackage{subcaption}
\usepackage{logicproof}
% \usepackage[backend=biber,style=numeric]{biblatex}
\usepackage[xetex]{hyperref}
\usepackage{fontspec}
\usetikzlibrary{arrows.meta}
\newcommand\emptyaccsupp[1]{\BeginAccSupp{ActualText={}}#1\EndAccSupp{}}
\setlength{\parindent}{2em}
\setmonofont{Consolas}
\setCJKmonofont{黑体}
% \setmainfont{Times New Roman}
\hypersetup{CJKbookmarks=true,colorlinks=true,citecolor=blue,%
            linkcolor=blue,urlcolor=blue,bookmarksnumbered=true,%
            bookmarksopen=true,breaklinks=true}
\lstset{
    % language = C,
    xleftmargin = 3em,xrightmargin = 3em, aboveskip = 1em,
	backgroundcolor = \color{white}, % 背景色
	basicstyle = \small\ttfamily, % 基本样式 + 小号字体
	rulesepcolor= \color{gray}, % 代码块边框颜色
	breaklines = true, % 代码过长则换行
	numbers = left, % 行号在左侧显示
	numberstyle = \small\emptyaccsupp, % 行号字体
    numbersep = -14pt, 
    keywordstyle=\color{purple}\bfseries, % 关键字颜色
    commentstyle =\color{red!50!green!50!blue!60}, % 注释颜色
    stringstyle = \color{red}, % 字符串颜色
    morekeywords={ASSERT, int64_t, uint32_t},
	frame = single, 
	showspaces = false, % 不显示空格
    showstringspaces = false,
	columns = fixed, % 字间距固定
    literate=
        {^-}{{{\color{black}\textbf{\color{red}-}}\colorbox{red!30}{\phantom{XX}}}}{1}
        {^+}{{{\color{black}\textbf{\color{green}+}}\colorbox{green!30}{\phantom{XX}}}}{1},
}

\raggedbottom

\title{\heiti\textbf{计算机逻辑基础}}
\author{第一次作业 \\ 
武泽恺 10225101429
}
\date{2025年9月20日}

\begin{document}
\maketitle

\subsection*{(1) $p \to q,\; r \to s \;\;\vdash\;\; (p \land r) \to (q \land s)$}

\begin{logicproof}{1}
  p \to q & Premise \\
  r \to s & Premise \\
  \begin{subproof}
    p \land r & Assumption \\
    p & $\land$e (3) \\
    r & $\land$e (3) \\
    q & $\to$e (1,4) \\
    s & $\to$e (2,5) \\
    q \land s & $\land$i (6,7)
  \end{subproof}
  (p \land r) \to (q \land s) & $\to$i (3--8)
\end{logicproof}

\subsection*{(2) $\neg(p \land q) \;\;\vdash\;\; \neg q \lor \neg p$}

\begin{logicproof}{1}
  \neg(p \land q) & Premise \\
  \begin{subproof}
    q & Assumption \\
    p & Assumption \\
    p \land q & $\land$i (3,2) \\
    \bot & $\neg$e (1,4) \\
    \neg p & $\neg$i (3--5) \\
    \neg q \lor \neg p & $\lor$i (6)
  \end{subproof}
  \begin{subproof}
    \neg q & Assumption \\
    \neg q \lor \neg p & $\lor$i (8)
  \end{subproof}
  \neg q \lor \neg p & $\lor$e (2--7,8--9)
\end{logicproof}

\subsection*{(3) $p \to (q \lor r),\; q \to s,\; r \to s \;\;\vdash\;\; p \to s$}

\begin{logicproof}{1}
  p \to (q \lor r) & Premise \\
  q \to s & Premise \\
  r \to s & Premise \\
  \begin{subproof}
    p & Assumption \\
    q \lor r & $\to$e (1,4) \\
    q & Assumption \\
    s & $\to$e (2,6) \\
    r & Assumption \\
    s & $\to$e (3,8) \\
    s & $\lor$e (5,6,8)
  \end{subproof}
  p \to s & $\to$i (4--10)
\end{logicproof}

\end{document}