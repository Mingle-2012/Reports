\documentclass{article}
\usepackage{fancyhdr}
\usepackage{ctex}
\usepackage{listings}
\usepackage[a4paper, body={18cm,22cm}]{geometry}
\usepackage{amsmath,amssymb,amstext,wasysym,enumerate,graphicx}
\usepackage{float,abstract,booktabs,indentfirst,amsmath}
\usepackage{multirow}
\usepackage{enumitem}
\usepackage{listings}
\usepackage{xcolor}
\usepackage{tabularx}
\usepackage[most]{tcolorbox}
\usepackage{accsupp}
\usepackage[bottom]{footmisc}
\usepackage{subcaption}
\usepackage{logicproof}
% \usepackage[backend=biber,style=numeric]{biblatex}
\usepackage[xetex]{hyperref}
\usepackage{fontspec}
\usetikzlibrary{arrows.meta}
\newcommand\emptyaccsupp[1]{\BeginAccSupp{ActualText={}}#1\EndAccSupp{}}
\setlength{\parindent}{2em}
\setmonofont{Consolas}
\setCJKmonofont{黑体}
% \setmainfont{Times New Roman}
\hypersetup{CJKbookmarks=true,colorlinks=true,citecolor=blue,%
            linkcolor=blue,urlcolor=blue,bookmarksnumbered=true,%
            bookmarksopen=true,breaklinks=true}
\lstset{
    % language = C,
    xleftmargin = 3em,xrightmargin = 3em, aboveskip = 1em,
	backgroundcolor = \color{white}, % 背景色
	basicstyle = \small\ttfamily, % 基本样式 + 小号字体
	rulesepcolor= \color{gray}, % 代码块边框颜色
	breaklines = true, % 代码过长则换行
	numbers = left, % 行号在左侧显示
	numberstyle = \small\emptyaccsupp, % 行号字体
    numbersep = -14pt, 
    keywordstyle=\color{purple}\bfseries, % 关键字颜色
    commentstyle =\color{red!50!green!50!blue!60}, % 注释颜色
    stringstyle = \color{red}, % 字符串颜色
    morekeywords={ASSERT, int64_t, uint32_t},
	frame = single, 
	showspaces = false, % 不显示空格
    showstringspaces = false,
	columns = fixed, % 字间距固定
    literate=
        {^-}{{{\color{black}\textbf{\color{red}-}}\colorbox{red!30}{\phantom{XX}}}}{1}
        {^+}{{{\color{black}\textbf{\color{green}+}}\colorbox{green!30}{\phantom{XX}}}}{1},
}

\raggedbottom

\title{\heiti\textbf{计算机逻辑基础}}
\author{第三次作业 \\ 
武泽恺 10225101429
}
\date{2025年10月24日}

\begin{document}
\maketitle

\section*{Completeness Proof Cases}

\begin{align*}
1.\quad & \Phi_1 \land \Phi_2 \vdash \Phi_1 \lor \Phi_2 \\[3pt]
&\text{证明:由合取消解得 } \Phi_1,\ \text{再由析取引入得 } \Phi_1 \lor \Phi_2。\\[6pt]
2.\quad & \Phi_1 \land \neg \Phi_2 \vdash \Phi_1 \lor \Phi_2 \\[3pt]
&\text{证明:由合取消解得 } \Phi_1,\ \text{再由析取引入得 } \Phi_1 \lor \Phi_2。\\[6pt]
3.\quad & \neg \Phi_1 \land \Phi_2 \vdash \Phi_1 \lor \Phi_2 \\[3pt]
&\text{证明:由合取消解得 } \Phi_2,\ \text{再由析取引入得 } \Phi_1 \lor \Phi_2。\\[6pt]
4.\quad & \neg \Phi_1 \land \neg \Phi_2 \vdash \neg(\Phi_1 \lor \Phi_2) \\[3pt]
&\text{证明:假设 } (\Phi_1 \lor \Phi_2),\ 
  \text{则由析取消解分两种情况:}\\
&\quad (i)\ \Phi_1 \Rightarrow \text{与 } \neg \Phi_1 \text{ 矛盾};\\
&\quad (ii)\ \Phi_2 \Rightarrow \text{与 } \neg \Phi_2 \text{ 矛盾}。\\
&\text{两种情况皆导致矛盾,因此 } \neg(\Phi_1 \lor \Phi_2)。\\[8pt]
5.\quad & \Phi_1 \land \Phi_2 \vdash \Phi_1 \to \Phi_2 \\[3pt]
&\text{证明:假设 } \Phi_1,\ \text{由合取消解得 } \Phi_2,\ 
  \text{由条件引入得 } \Phi_1 \to \Phi_2。\\[6pt]
6.\quad & \Phi_1 \land \neg \Phi_2 \vdash \neg(\Phi_1 \to \Phi_2) \\[3pt]
&\text{证明:由条件的等价式 } \Phi_1 \to \Phi_2 \equiv \neg \Phi_1 \lor \Phi_2,\\
&\text{得其否定为 } \neg(\Phi_1 \to \Phi_2) \equiv \Phi_1 \land \neg \Phi_2。\\
&\text{因此从前提即可直接得到结论。}\\[6pt]
\end{align*} 

\begin{align*}
7.\quad & \neg \Phi_1 \land \Phi_2 \vdash \Phi_1 \to \Phi_2 \\[3pt]
&\text{证明:假设 } \Phi_1,\ \text{则 } \Phi_2 \text{ 已由前提成立},\ 
  \text{因此 } \Phi_1 \to \Phi_2。\\[6pt]
8.\quad & \neg \Phi_1 \land \neg \Phi_2 \vdash \Phi_1 \to \Phi_2 \\[3pt]
&\text{证明:在经典逻辑中,当前件 } \Phi_1 \text{ 为假时,} 
  \Phi_1 \to \Phi_2 \text{ 恒真。}\\
&\text{因此由 } \neg \Phi_1 \text{ 立即推出 } \Phi_1 \to \Phi_2。
\end{align*}

\section*{Conversion to CNF}

\[
\Phi = \neg ( r \land ( \neg ( q \to (\neg p \to (q \land r)) ) ) )
\]

$$
= CNF((NNF(IMPL\_FREE(\Phi))))
$$

\textbf{Step 1: IMPL\_FREE}
\[
IMPL\_FREE(\neg ( r \land ( \neg ( q \to (\neg p \to (q \land r)) ) ) ))
\]
\[
= \neg ( r \land ( \neg ( \neg q \lor (p \lor (q \land r)) ) ) )
\]

\textbf{Step 2: NNF}
\[
NNF(\neg ( r \land ( \neg ( \neg q \lor (p \lor (q \land r)) ) ) ))
\]
\[
= \neg r \lor \neg ( \neg ( \neg q \lor (p \lor (q \land r)) ) )
\]
\[
= \neg r \lor ( \neg q \lor (p \lor (q \land r)))
\]

\textbf{Step 3: CNF}

\[
= CNF(\neg r \lor ( \neg q \lor (p \lor (q \land r))))
\]
\[
= DISTR(CNF(\neg r), CNF ( \neg q \lor (p \lor (q \land r))))
\]
\[
= DISTR(\neg r, DISTR(\neg q, CNF(p \lor (q \land r))))
\]
\[
= DISTR(\neg r, DISTR(\neg q, DISTR(p, q \land r)))
\]
\[
= DISTR(\neg r, DISTR(\neg q, (p \lor q) \land (p \lor r)))
\]
\[
= DISTR(\neg r, (\neg q \lor p \lor q) \land (\neg q \lor p \lor r))
\]
\[
= (\neg r \lor \neg q \lor p \lor q) \land (\neg r \lor \neg q \lor p \lor r)
\]
\end{document}