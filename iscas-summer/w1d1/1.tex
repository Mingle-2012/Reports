\documentclass[UTF8]{ctexart}
\usepackage[a4paper,top=2.54cm,bottom=2.54cm,left=3.17cm,right=3.17cm]{geometry}
\usepackage{graphicx}
\usepackage{amsmath, amssymb}
\usepackage{hyperref}
\usepackage{fancyhdr}
\usepackage{titlesec}

\fancypagestyle{plain}{
    \fancyhf{}
    \fancyhead[L]{中国科学院大学软件所夏令营科研实习线上交流总结报告}
    \fancyhead[R]{\thepage}
    \renewcommand{\headrulewidth}{0.4pt}
}

\pagestyle{plain}

\titleformat{\section}{\Large\bfseries}{\thesection}{1em}{}
\titleformat{\subsection}{\large\bfseries}{\thesubsection}{1em}{}

\hypersetup{
    colorlinks=true,
    linkcolor=blue,
    urlcolor=blue
}


\title{\heiti 科研实习第一次线上交流总结报告}
\author{武泽恺\footnote{华东师范大学,软件工程学院,本科三年级学生在读,email: zekaiwu@stu.ecnu.edu.cn}}
\date{2025年7月22日}

\begin{document}

\maketitle

\section{引言}
在本次夏令营中,我有幸参与了中国科学院大学软件所的科研实习项目。本次实践所选择的主题是“向量数据库系统自动化测试用例生成方法”,旨在通过自动化测试提升向量数据库系统的可靠性和性能。

\section{讨论内容}

我们首先讨论了两周的科研实习安排,包括实践内容、交流方式以及预期成果。

实践内容方面,在未来的两周时间里,我将主要研读两篇重点论文(\textit{Towards Reliable Vector Database Management Systems: A Software Testing Roadmap for 2030}\textsuperscript{1} 和 \textit{Toward Understanding Bugs in Vector Database Management Systems}\textsuperscript{2}),从用户的角度去看待向量数据库系统中可能存在的问题,并重点总结第二篇文章的工作和贡献,以PPT的形式撰写一篇总结报告。

交流方式方面,我们将通过线上会议的方式进行定期交流。交流结束后,撰写一篇讨论总结,并在第二天中午之前反馈给老师,以便于加深讨论内容的理解和记忆。下一次交流定于7月25日,届时讨论对论文的理解。

预期成果方面,本次科研实践将作为一个探索性的尝试,主要目的是增进师生之间的交流,积累科研经验。其次,在阅读并总结了相关文献后,需要针对文章中所发现的某一类问题来设计一个测试方法,实现一个初步的版本。同时,这次的课题也可以成为未来科研工作的一个研究方向。

\section{未来计划}

在接下来的两周中,我将按照以下计划进行:

\begin{itemize}
    \item 第一周 (7.22-7.25):
    \begin{itemize}
        \item 研读两篇论文,重点总结第二篇论文\textsuperscript{2}的工作和贡献。
        \item 将总结的内容汇总成第二次交流会议的PPT。
        \item 7.25日(星期五)进行第二次交流会议,讨论对论文的理解和总结,并在会议后撰写讨论总结。
    \end{itemize}
    \item 第二周 (7.28-8.1):
    \begin{itemize}
        \item 设计一个针对第二篇论文中所发现某一个具体问题的测试方法。
        \item 选择一个知名度高、开源的向量数据库系统,将我们的方法实现一个初步的版本,并测试其有效性。
        \item 我们将进行第三次和第四次交流会议,定期汇报进展和讨论遇到的问题。
    \end{itemize}
\end{itemize}

\end{document}
