\documentclass[UTF8,openany]{ctexbook}

% 论文版面要求:
% 统一按 word 格式A4纸(页面设置按word默认值)编排、打印、制作。
% 正文内容字体为宋体;字号为小4号;字符间距为标准;行距为25磅(约0.88175cm)。

%%%%% ===== 页面设置
\usepackage[a4paper,top=2.54cm,bottom=2.54cm,left=3.17cm,right=3.17cm,%
            ]{geometry}
\usepackage{tcolorbox}
\usepackage{colortbl}
\usepackage{dirtree}
\usepackage{longtable}
\usepackage{booktabs}
            
\setlength{\parindent}{2em}
%默认的弹性间距会导致文中某些排版flush的时候,出现大量空白。
\setlength{\parskip}{0.5em} %指定固定段后间距,默认为弹性间距。
\setlength{\intextsep}{10pt} %固定浮浮动体前后间距。
\usepackage{enumitem}
\usepackage{ulem}

%%%%% =====章节 标题 设置
\ctexset{%
  contentsname={\vspace{-3.5em}\centerline{\zihao{-3}\heiti 目\quad 录}\vspace{-0.7em}},
  listfigurename={\vspace{-3.5em}\centerline{\zihao{-3}\heiti 插\ 图\ 目\ 录}\vspace{-0.5em}},
  listtablename={\vspace{-3.5em}\centerline{\zihao{-3}\heiti 表\ 格\ 目\ 录}\vspace{-0.5em}},
  bibname={\vspace{-3em}\centerline{\zihao{-3}\heiti 参\ 考\ 文\ 献}\vspace{3em}},
  chapter={name={,},
  number=\arabic{chapter}, %指定章序号为一二三。。。。
  nameformat={\zihao{-2}\bfseries},
  titleformat={\zihao{-2}\bfseries},
  format=\raggedright,
  beforeskip={10pt},
  afterskip={10pt},
  pagestyle={fancy}
  },
section={format=\raggedright,
  nameformat={\zihao{4}\bfseries},
  titleformat={\zihao{4}\bfseries},
%           afterskip={1ex plus 0.2ex}
  beforeskip={1ex},% 固定段前段后间距,
  afterskip={1ex}
  },
subsection={format=\raggedright,
  nameformat={\zihao{-4}\bfseries},
  titleformat={\zihao{-4}\bfseries},
%           afterskip={0.5ex plus 0.1ex}
  beforeskip={0.5ex},
  afterskip={0.5ex}
  }
}
%%%%% ===== 中英文字体
%\setsansfont{Myriad Pro} % 无衬线字体 sans serif \sffamily
%\setmonofont{Consolas}   % 等宽字体 typewriter \ttfamily
%\newcommand{\Times}{\fontspec{Times New Roman}}
%% 中文字体
%\setCJKmainfont[BoldFont={Microsoft YaHei},ItalicFont={KaiTi}]{NSimSun}
%\setCJKsansfont{Microsoft YaHei}
%\setCJKmonofont{KaiTi}
% \setCJKfamilyfont{STSong}{方正小标宋_GBK}\newcommand{\STSong}{\CJKfamily{STSong}}
\setCJKfamilyfont{songti}{STZhongsong}\newcommand{\STSong}{\CJKfamily{STSong}}

%%%%% ===== 常用宏包
\usepackage{amsmath,amssymb,amsfonts,bm}
\usepackage[amsmath,thref,thmmarks,hyperref]{ntheorem}
\usepackage{graphicx,xcolor,float}
\usepackage{fancyhdr}
\usepackage{tocloft} % 设置目录中的条目间距


\renewcommand\cftdot{\textsubscript{……}}
\renewcommand\cftdotsep{0}

\setlength{\cftbeforechapskip}{1pt}
\renewcommand{\cftchapleader}{\cftdotfill{\cdot}}


\usepackage{booktabs} % toprule, midrule, bottomrule
\usepackage{varwidth} % 可变宽度的 parbox

%%%%% ===== 参考文献与链接
\usepackage[numbers,sort&compress,sectionbib,super, square]{natbib} %引用上标,禁用连续缩写。
\newcommand{\upcite}[1]{\textsuperscript{\cite{#1}}}


\usepackage[xetex,pagebackref]{hyperref}
\hypersetup{CJKbookmarks=true,colorlinks=true,citecolor=blue,%
            linkcolor=blue,urlcolor=blue,bookmarksnumbered=true,%
	        bookmarksopen=true,breaklinks=true}
	        
	        
	        
\iffalse   % 调试时,可去掉,以用于显示引用位置。
\renewcommand*{\backrefalt}[4]{%
\ifcase #1 No citations.%
\or Cited on page #2.%
\else Cited on pages #2.%
%\else #1 Cited on pages #2.%
\fi
}

\else
\renewcommand*{\backrefalt}[4]{}
\fi

%%%%% ===== 浮动图表的标题
\usepackage[margin=2em,labelsep=space,skip=0.5em,font=normalfont]{caption}
\DeclareCaptionFormat{mycaption}{{\heiti\color{blue} #1}#2{\kaishu #3}}
\captionsetup{format=mycaption,tablewithin=chapter,figurewithin=chapter}%,belowskip=-10pt
%\setlength{\belowcaptionskip}{-10pt}

%%%%%% ===== 浮动图表的比例默认50%以下,否则无法浮动。
\renewcommand\floatpagefraction{.9} %当浮动体小于页面90%时进行直接放置。
\renewcommand\topfraction{.9}  
\renewcommand\bottomfraction{.9}  
\renewcommand\textfraction{.1}



%%%%% ===== 算法
\usepackage{algorithm,algpseudocode}

%%%%% ===== 其他
\usepackage{ulem}
\def\ULthickness{1pt}




%%%%%===== Code Style代码
\usepackage{listings}
\usepackage{color}

\definecolor{dkgreen}{rgb}{0,0.6,0}
\definecolor{gray}{rgb}{0.5,0.5,0.5}
\definecolor{mauve}{rgb}{0.58,0,0.82}

\usepackage{accsupp}



\newcommand\emptyaccsupp[1]{\BeginAccSupp{ActualText={}}#1\EndAccSupp{}}

\lstset{
    % language = C,
    showstringspaces=false,
    xleftmargin = 3em,xrightmargin = 3em, aboveskip = 1em,
	backgroundcolor = \color{white}, % 背景色
	basicstyle = \small\ttfamily, % 基本样式 + 小号字体
	rulesepcolor= \color{gray}, % 代码块边框颜色
	breaklines = true, % 代码过长则换行
	numbers = left, % 行号在左侧显示
	numberstyle=\emptyaccsupp,
    numbersep = 14pt, 
    keywordstyle=\color{purple}\bfseries, % 关键字颜色
    commentstyle =\color{red!50!green!50!blue!60}, % 注释颜色
    stringstyle = \color{red}, % 字符串颜色
    morekeywords={ASSERT, int64_t, uint32_t},
	% frame = shadowbox, % 用(带影子效果)方框框住代码块
	frame = single, % 用(带影子效果)方框框住代码块
	showspaces = false, % 不显示空格
	columns = fixed, % 字间距固定
  framesep=1em
} 
\lstset{
    sensitive=true,
    moreemph={ASSERT, NULL}, emphstyle=\color{red}\bfseries,
    moreemph=[2]{int64_t, uint32_t, tid_t, uint8_t, int16_t, uint16_t, int32_t, size_t}, emphstyle=[2]\color{purple}\bfseries,
    showspaces = false, % 不显示空格
    }



\newcommand{\mcc}[1]{\multicolumn{1}{c}{\underline{\makebox[10em][c]{#1}}}}
\newcommand{\mce}[1]{\multicolumn{1}{c}{\underline{\makebox[15em][l]{#1}}}}


\pagestyle{fancy}
\fancyhf{}  % 清除以前对页眉页脚的设置

\newcommand{\makeheadrule}{%% 定义页眉与正文间双隔线
    \makebox[0pt][l]{\rule[.7\baselineskip]{\headwidth}{0.3pt}}%0.4
    \rule[0.85\baselineskip]{\headwidth}{1.0pt}\vskip-.8\baselineskip}
\makeatletter
\renewcommand{\headrule}{%
    % {\if@fancyplain\let\headrulewidth\plainheadrulewidth\fi\makeheadrule}}
    {\makeheadrule}}
\makeatother
\renewcommand{\chaptermark}[1]{\markboth{\CTEXthechapter \ #1}{}}
\renewcommand{\sectionmark}[1]{\markright{\thesection \ #1}{}}
%\fancyhead[RO,LE]{{\small\songti\rightmark}}     % 节标题
%\fancyhead[RE]{{\small\songti\leftmark}}      % 章标题
\fancyhead[C]{《中国文化通论》课程期末论文}
% \fancyhead[RO,LE]{$\cdot$ {\small\thepage} $\cdot$}
\fancyfoot[C]{{-\thepage-}}
%\fancyfoot[CO,CE]{{\thepage}}

\ctexset{chapter/break={}}

\begin{document}

\begin{titlepage}
    \begin{center}

        {
            \begin{figure}[H]
                \vspace{5cm}
                \includegraphics[width=14cm]{img/0.png}
            \end{figure}
            \heiti\zihao{2}《中国文化通论》课程期末论文\\
            \vspace{1.8em}
            
        }
        
        \zihao{-3}
        \begin{tabular}{p{0cm}p{0em}@{\extracolsep{0.5ex}}cc}
            ~ & \hfill             &  & \mcc{武泽恺\quad 10225101429}      \\
        \end{tabular}
        \\[8em]
        \zihao{-2}2024年6月
    \end{center}
    \thispagestyle{fancy}
    \fancyfoot[C]{}
\end{titlepage}
\fancyfoot[C]{-\thepage-}

\setcounter{page}{1}
\pagenumbering{roman}

\thispagestyle{fancy}
\newpage

\setcounter{page}{1}
\pagenumbering{arabic}

\begin{center}
  \large \textbf{无字碑歌} \cite{ref1}
\end{center}

万岁通天元年的长夏,磅礴大雨从阴霾沉沉的天空中倾盆而下,重叠的雨幕遮住了视线,为整个洛阳城披上一层深邃的神秘面纱。地上的雨水横流,洗涤着唐王朝百年的历史尘埃与痕迹,等待着新的篇章被书写。

整个庞大的唐帝国仿佛都在这场突如其来的大雨中按下了暂停键。洛阳城内失去了往日的人声鼎沸,取而代之的是纷乱的雨声,充斥着神都的每一个角落。但是,唯有一处却灯火通明,弦歌不辍——明堂。

明堂的大门高耸入云,两侧有雄伟的金狮镇守,金色的外表在雨水的冲刷下显得格外明亮。大门虚掩着,龙脑香的气息混杂着有序的人声从门缝中渗出,被无限放大着。象征着皇帝无上权力的宝座高居大殿中央,金制的椅背和典型的须弥座底,让皇帝的形象尊贵而神格化,座上空无一人。不同于前朝或现唐以往皇帝的金銮宝殿,宝座后面并非常见的雕龙髹金屏风,取而代之的是一尊夹纻佛像,其体型之巨大,手尚可容纳数十人。整个明堂,是当今女皇武则天集各国君臣聚钱亿万,买尽天下铜铁,历时一年有余,终于建成。\cite{ref2}

想必御台旁站着的就是武则天了,一代天骄女皇,虽已年过古稀,满头华发下是未老的容颜,眼里炯炯有神。即使是九五之尊,或许脱去紫金凤袍后看上去与宫人无异。此时的她,正静候一个人的到来。

这是六年里第七十四次听取经义讲授,武则天暗自思忖道。

即位以来,她下诏规定佛教列于道教之上,僧尼位于道士女冠之前。尊佛拜佛,以佛训政,翻译佛典,教化万方。每月月初,她都会请名寺方丈前来明堂,为她亲自讲授佛法。法藏是明堂的常客,也是武则天深深敬仰的大师,钦定“贤首国师“。

武则天在她的人生中第一次接触佛学,已经是一甲子之前的事情了。其母亲杨氏和隋朝宗室算的上是血缘近亲,隋朝历代君主笃信佛家,杨氏也是如此。从小,武则天随母亲听取讲经,探求佛法,拜佛烧香,抄录经文。抄录的经文主要以《兜沙经》、《法华经》为主。杨氏还曾于家中设立佛堂,供拜观音,收纳佛经,筑架以藏。故武则天深受母亲所带来的佛教洗礼影响。\cite{ref2}她的父亲武士彟深受杨氏影响,对释家佛道逐渐熟悉入迷。士彟还曾与唐太宗李世民一同在长沙寺拜阿育王像。家庭的熏陶让她有更多的机会接触到佛学,也在她的内心之中种下了深深的情愫。在她进入宫闱之前,曾一度有出家为尼的想法。

陛下,贤首国师到了。传话的内侍推开了偏殿的侧门,两眼望向地面疾步走来。

宣。

宣贤首国师上殿。内侍面朝正南方,提高了嗓音。

身披袈裟,手持禅杖,一长者从偏门入,徐徐走来。其步伐沉稳而有力,混杂着渐弱的雨声,周围的一切都变得宁静而庄严。其目光深邃,就好像,能够洞察世间的一切。

武则天所等之人名曰法藏,乃华严宗始祖。华严宗乃大乘佛教一大宗派,因尊《华严经》为最高经典得名,武则天正是大乘佛教的信仰者,更是华严宗的虔诚信徒。\cite{ref4}

就在上一次,法藏为武则天说明了真如本体,讲述了无生无灭。无生无灭讲求世间万物都不会自然合而生,亦不会自然归于无,一切不过是本体所产生的幻象。

皇权也是如此。当上大唐的天后也好,革唐代周的女皇也罢,不过是冕旒一顶,凤袍一件,不过是繁华过眼,红尘一梦,不过是流光一瞬,空幻如烟。珍馐美味,荣华富贵,生杀予夺,陟罚臧否,都是皇权所带来的现象罢了。
只是,在追求皇权的道路上,武则天已经付出太多了。四十年前的一个雨夜,王皇后登门拜访,看望武则天刚刚诞下不久的女儿。王皇后离去后,得知皇帝很快赶来,不肯错过任何一个良机的她狠下心来,亲手掐住亲生女儿的脖子,待到女儿没有呼吸,她不敢看女儿一眼,只是又慢慢盖上了被子。等到皇帝发现女儿已经不在人世时,嫁祸王皇后杀死了她。\cite{ref5}

那一夜,她哭了,哭的很孤独,没有任何人知道。窗外狂风咆哮着,灯笼急速地摇晃着,地面上充满了红色的碎片。王皇后得废,武则天顺利上位,长孙无忌为首的陇西望族失去靠山,最大的门阀得以肃清,最大的对手得以消灭,再也没有人能够阻止她,她需要做的只是再往前迈一步。她知道,彻底踏上了追逐皇权的道路,每一步都会充满了血与泪,在这条满是悲哀的路上,首先要学会独自撑伞——即使风雨再大,也要前进,前进。 

老衲今日为陛下讲解六相。\cite{ref6}

思绪,即使是千丝万缕,最终也飘回了这里。法藏落座。

所谓六相,乃事物之六类相状。

总相,别相,同相,异相,成相与坏相。吾辈之修行可以六相分之,凡世间所有之事物,亦不过如此六类相状。一切缘起之现象,也具足六相。六相彼此圆融无碍,是曰六相圆融。

大师所说,不甚理解。

老衲仍以殿外金狮为比,为陛下阐述其中之道。

陛下请看,狮子是总相,即视整个狮子为一整体,名为总相。

就是说这整个狮子本身是总相。

陛下说的是。而五根差别是别相。即狮子之五官差异,名曰别相。

也就是说狮子的各部分是别相。

陛下所言极是。而共从一缘起是同相。即狮子之五官皆是因缘而生起狮子,是各部分的共同之处,名曰同相。

那么各部分因缘而合,因合而成,是同相。

陛下解读的是。而眼、耳等不相滥是异相。即狮子之五官各不相同,是各部分的差异之处,名曰异相。

各部分又不相同,所以叫异相。

正是。而诸根合会有狮子是成相。即狮子之五官共同组合形成狮子,是部分组合而形成整体,故名曰成相。

因缘形成一个事物,是成相。

陛下说的是。而诸根各住自位是坏相。即狮子之五官各自独立,各守其位,不为整体,是故名曰坏相。

各部分始终分离,所以叫做坏相。

正是。是以“六相圆融”者,乃诸法之真理,世间万象之精髓。老衲以金狮作比,狮之总相,统摄全局,正如陛下之于国家,代天巡狩,一统四方。狮之五官,各具别相,正如百官之于陛下,各司其职,各有所长。狮之五官虽异,然共由狮体所生,此乃同相,正如百官之于国家,皆为国之栋梁,共辅国事。五官各不尽相同,各有其用,是为异相,正如百官之于大周,各守其位,不相逾越。狮之五官,共成狮形,此乃成相,正如万民之于国家,万民和谐,百官协作,方能国泰民安。五官各守其位,不相混淆,为坏相之体现,此坏非真坏,正如百官万民各有所守,各安其分,国家之基始得稳固。

老衲以为,六者相互依存,相互制约,圆融无碍,共构狮之全体,此乃“六相圆融”之真谛。若以此观天下,则万物皆备此六相,圆融无碍,和谐共生,此乃天地之道也。

世间万象,皆如金狮。大师之言,深入浅出,朕受益匪浅。

武则天最感触的还是佛家的包容。世间一切万物可以六相化之,那么天下人看来水火不容的儒释道三家,也可以有圆融一体的契机。对于儒家黄老之学,如果定要有个高低喜恶之分,她更倾向道家。道家讲求道法自然,无为而治,顺天时而为,她的许多政策无不取自其中之道。

反观儒学,甚至有一丝厌恶之情。儒家的三从四德,早已筹划好了每一个女人的终生。儒家强调男尊女卑,不许妇人与闻国政,\cite{ref3}注定了武则天要通过非常人的手段来得到权力。一人侍奉唐太宗、唐高宗二主,亲手杀死女儿,武则天是一个十足的狠人,更是一个争强好胜的人。儒家越不让她做什么,她就一定拼足了勇气去挑战那不可逾越的底线。即便如此,在崇佛的同时,道教和儒教也没有被过分的贬低。因为佛法使然,普渡万生的大乘哲理,造就了武则天强大的包容心。儒释道乃世间万物之一,亦可以六相化之。在尊重六相俱存的基础上,寻求六相共同点和互补之处,不正是法藏所表达的哲理吗?

朕观天下,百官各有其职,各守其位,共辅朕躬,以成国家之繁荣。朕当以总相之心,统御四方,以别相之智,选拔贤能,以同相之情,凝聚人心,以异相之明,洞察秋毫,以成相之力,致国家发展,以坏相之定,使秩序井然。六相圆融,和谐共生。朕当以此为鉴,谋国家之盛世,成万世之基业。

殿外雨渐歇。微风拂过,湿土的气息涌入明堂。雨洼中浮起金狮子的流光,比任何以往都更加闪亮。

大乘佛法培养了武则天博大的胸怀。她深知她的天下,不是她和门阀贵族的天下,而是天下人的天下。

五十多年的参政与执政生涯里,武则天开科举,启殿试,得以有君子满朝;减赋税,兴农桑,使百姓安居乐业;扬佛法,倡慈悲,万国来华取佛经;安边疆,御敌寇,护佑国土安宁。选贤举能,政治清明;农业繁盛,国富民强;普及教育,文化昌盛;拓展外交,修好万邦。今日之大周,远胜昔日贞观之大唐。

今日去西安还能参观唐高宗李治与武则天合葬的乾陵。陵前并立着两块巨大的石碑,西侧的一块名曰述圣碑,黑漆碑面,字填金粉,光彩照人。这是武则天为高宗歌功颂德而立的碑,她还亲自撰写了数千字的碑文。 

东侧的是武则天的无字碑。无字碑正如其名所说一般,一字皆无。

有人说,她认为文字无法完全表达她的功德。

也有人说,她自知罪孽深重,无言以对天下人。

一切功过交由后人评说。

\newpage

\begin{thebibliography}{1}

	\bibitem{ref1}
	文章标题参考影视作品, {\it{无字碑歌:武则天传}}. 陈燕民(导演), 张文华(编剧), 李忆嘉(编剧), 2006.

  \bibitem{ref2}
  李, 金松. 述学校笺. {\it{中国历史文集丛刊}}. 中华书局, 北京, 2014.

  \bibitem{ref3}
	陈寅恪, {\it{金明馆丛稿二编}}. 上海古籍出版社, 上海, 2020.

  \bibitem{ref4}
	张祥龙, {\it{拒秦兴汉和应对佛教的儒家哲学:从董仲舒到陆象山}}. 广西师范大学出版社, 桂林, 2012.

  \bibitem{ref5}
  胡戟, {\it{武则天本传}}. 北京大学出版社, 北京, 2011.

  \bibitem{ref6}
	方立天释译, 星云大师监修, {\it{华严金师子章}}. 东方出版社, 北京, 2016.

  
  

\end{thebibliography}

\end{document}