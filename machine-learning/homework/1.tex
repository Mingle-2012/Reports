\documentclass{article}
\usepackage{fancyhdr}
\usepackage{ctex}
\usepackage[a4paper, body={18cm,22cm}]{geometry}
\usepackage{amsmath,amssymb,amstext,wasysym,enumerate,graphicx}
\usepackage{float,abstract,booktabs,indentfirst,amsmath}
\usepackage{multirow}
\usepackage{enumitem}
\usepackage{xcolor}
\usepackage{tabularx}
\usepackage[most]{tcolorbox}
\usepackage{accsupp}
\usepackage[bottom]{footmisc}
\usepackage{subcaption}
\usepackage{logicproof}
% \usepackage[backend=biber,style=numeric]{biblatex}
\usepackage[xetex]{hyperref}
\usepackage{fontspec}
\usepackage{listingsutf8}
\usepackage{xcolor}
\usetikzlibrary{arrows.meta}
\newcommand\emptyaccsupp[1]{\BeginAccSupp{ActualText={}}#1\EndAccSupp{}}
\setlength{\parindent}{2em}
\setmonofont{Consolas}
\setCJKmonofont{黑体}
% \setmainfont{Times New Roman}
\hypersetup{CJKbookmarks=true,colorlinks=true,citecolor=blue,%
            linkcolor=blue,urlcolor=blue,bookmarksnumbered=true,%
            bookmarksopen=true,breaklinks=true}
\lstset{
    % language = C,
    inputencoding=utf8,
    extendedchars=false,
    showstringspaces=false,
    xleftmargin = 3em,xrightmargin = 3em, aboveskip = 1em,
	backgroundcolor = \color{white}, % 背景色
	basicstyle = \small\ttfamily, % 基本样式 + 小号字体
	rulesepcolor= \color{gray}, % 代码块边框颜色
	breaklines = true, % 代码过长则换行
	numbers = left, % 行号在左侧显示
	numberstyle=\emptyaccsupp,
    numbersep = 14pt, 
    keywordstyle=\color{purple}\bfseries, % 关键字颜色
    commentstyle =\color{red!50!green!50!blue!60}, % 注释颜色
    stringstyle = \color{red}, % 字符串颜色
    morekeywords={ASSERT, int64_t, uint32_t},
	% frame = shadowbox, % 用(带影子效果)方框框住代码块
	frame = single, % 用(带影子效果)方框框住代码块
	showspaces = false, % 不显示空格
	columns = fixed, % 字间距固定
  framesep=1em
} 
\lstset{
    sensitive=true,
    moreemph={ASSERT, NULL}, emphstyle=\color{red}\bfseries,
    moreemph=[2]{int64_t, uint32_t, tid_t, uint8_t, int16_t, uint16_t, int32_t, size_t}, emphstyle=[2]\color{purple}\bfseries,
    showspaces = false, % 不显示空格
    }

\raggedbottom

\begin{document}

\boxed{
\begin{aligned}
\mathbf{w} &= (1,1)^T, \\
b &= -4, \\
\text{超平面:} &  x_1 + x_2 - 4 = 0,\\
\text{决策函数:} & f(\mathbf{x}) = \operatorname{sign}(x_1 + x_2 - 4)
\end{aligned}
}
\begin{tikzpicture}[scale=1.2,>=stealth]

% 坐标轴
\draw[->] (0,0) -- (5.2,0) node[right]{$x_1$};
\draw[->] (0,0) -- (0,5.2) node[above]{$x_2$};

% 超平面和间隔边界
\draw[thick, blue] (0,4) -- (4,0) ;
\draw[dashed, gray] (0,3) -- (3,0) ;
\draw[dashed, gray] (0,5) -- (5,0) ;

% 正例 (+1)
\filldraw[red] (1,2) circle (2.2pt);
\filldraw[red] (2,3) circle (2.2pt);
\filldraw[red] (3,3) circle (2.2pt);

% 负例 (-1)
\filldraw[blue] (2,1) circle (2.2pt);
\filldraw[blue] (3,2) circle (2.2pt);

% 样本标签(偏移放置避免重叠)
\node[red, anchor=east] at (0.9,2.1) {$\mathbf{x}_1$};
\node[red, anchor=south east] at (2,3.1) {$\mathbf{x}_2$};
\node[red, anchor=south west] at (3.1,3.1) {$\mathbf{x}_3$};
\node[blue, anchor=north east] at (1.9,0.9) {$\mathbf{x}_4$};
\node[blue, anchor=north west] at (3.1,2) {$\mathbf{x}_5$};

% 支持向量标记(箭头稍微偏移)
\draw[->, thick] (1,2.4) -- (1,2.15);
\draw[->, thick] (3,2.4) -- (3,2.15);
\node[black, font=\small] at (1,2.55) {支持向量};
\node[black, font=\small] at (3,2.55) {支持向量};

% 图例
\begin{scope}[shift={(3.6,4.3)}]
  \draw[red, fill=red] (0,0) circle (2pt);
  \node[right] at (0.2,0) {正例};
  \draw[blue, fill=blue] (0,-0.4) circle (2pt);
  \node[right] at (0.2,-0.4) {反例};
  \draw[blue, thick] (-0.1,-0.8) -- (0.4,-0.8);
  \node[right] at (0.5,-0.8) {超平面 $x_1 + x_2 - 4=0$};
  \draw[dashed, gray] (-0.1,-1.2) -- (0.4,-1.2);
  \node[right] at (0.5,-1.2) {间隔边界};
\end{scope}

\end{tikzpicture}

\begin{table}[h!]
\centering
\begin{tabular}{c|cc}
\hline
\textbf{真实类别} & \textbf{预测为正例 (Pred +)} & \textbf{预测为反例 (Pred -)} \\
\hline
\textbf{正例 (True +)} & 200 (TP) & 100 (FN) \\
\textbf{反例 (True -)} & 200 (FP) & 500 (TN) \\
\hline
\end{tabular}
\caption{模型的混淆矩阵}
\end{table}



\end{document}