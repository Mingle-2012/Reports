\documentclass{article}
\usepackage{fancyhdr}
\usepackage{ctex}
\usepackage[a4paper, body={18cm,22cm}]{geometry}
\usepackage{amsmath,amssymb,amstext,wasysym,enumerate,graphicx}
\usepackage{float,abstract,booktabs,indentfirst,amsmath}
\usepackage{multirow}
\usepackage{enumitem}
\usepackage{xcolor}
\usepackage{tabularx}
\usepackage[most]{tcolorbox}
\usepackage{accsupp}
\usepackage[bottom]{footmisc}
\usepackage{subcaption}
\usepackage{logicproof}
% \usepackage[backend=biber,style=numeric]{biblatex}
\usepackage[xetex]{hyperref}
\usepackage{fontspec}
\usepackage{listingsutf8}
\usepackage{xcolor}
\usetikzlibrary{arrows.meta}
\newcommand\emptyaccsupp[1]{\BeginAccSupp{ActualText={}}#1\EndAccSupp{}}
\setlength{\parindent}{2em}
\setmonofont{Consolas}
\setCJKmonofont{黑体}
% \setmainfont{Times New Roman}
\hypersetup{CJKbookmarks=true,colorlinks=true,citecolor=blue,%
            linkcolor=blue,urlcolor=blue,bookmarksnumbered=true,%
            bookmarksopen=true,breaklinks=true}
\lstset{
    % language = C,
    inputencoding=utf8,
    extendedchars=false,
    showstringspaces=false,
    xleftmargin = 3em,xrightmargin = 3em, aboveskip = 1em,
	backgroundcolor = \color{white}, % 背景色
	basicstyle = \small\ttfamily, % 基本样式 + 小号字体
	rulesepcolor= \color{gray}, % 代码块边框颜色
	breaklines = true, % 代码过长则换行
	numbers = left, % 行号在左侧显示
	numberstyle=\emptyaccsupp,
    numbersep = 14pt, 
    keywordstyle=\color{purple}\bfseries, % 关键字颜色
    commentstyle =\color{red!50!green!50!blue!60}, % 注释颜色
    stringstyle = \color{red}, % 字符串颜色
    morekeywords={ASSERT, int64_t, uint32_t},
	% frame = shadowbox, % 用(带影子效果)方框框住代码块
	frame = single, % 用(带影子效果)方框框住代码块
	showspaces = false, % 不显示空格
	columns = fixed, % 字间距固定
  framesep=1em
} 
\lstset{
    sensitive=true,
    moreemph={ASSERT, NULL}, emphstyle=\color{red}\bfseries,
    moreemph=[2]{int64_t, uint32_t, tid_t, uint8_t, int16_t, uint16_t, int32_t, size_t}, emphstyle=[2]\color{purple}\bfseries,
    showspaces = false, % 不显示空格
    }

\raggedbottom

\title{\heiti\textbf{机器学习实践报告}}
\author{第二次实验 \\ 
武泽恺 10225101429
}
\date{2025年11月8日}



\begin{document}
\maketitle

\section{实验目的}
\begin{itemize}
    \item 理解线性回归的基本原理;
    \item 学习Scikit-learn的使用;
    \item 掌握模型训练、预测和评估流程。
\end{itemize}

\section{实验环境}
\begin{itemize}
    \item Python 3.11.7
    \item 主要库:
    \begin{itemize}
        \item \texttt{scikit-learn}
        \item \texttt{numpy}
        \item \texttt{matplotlib}
    \end{itemize}
\end{itemize}

\section{实验内容与步骤}

\subsection{导入模块}
使用 \texttt{LinearRegression}、\texttt{mean\_squared\_error} 等模块:
\begin{lstlisting}[language=Python]
from sklearn.linear_model import LinearRegression
from sklearn.metrics import mean_squared_error, r2_score
\end{lstlisting}

\subsection{生成数据}
生成符合线性关系 $y = -5x + 0.1$ 的模拟数据,并加入噪声:
\begin{lstlisting}[language=Python]
X = 2 * np.random.rand(100, 1)
y_true = -5 * X + 0.1
noise = np.random.randn(100, 1) * 0.5
y = y_true + noise
\end{lstlisting}

\subsection{构建模型}
使用Scikit-learn提供的线性回归模型:
\begin{lstlisting}[language=Python]
model = LinearRegression()
\end{lstlisting}

\subsection{模型训练}
\begin{lstlisting}[language=Python]
model.fit(X, y)
\end{lstlisting}
由于Scikit-learn的LinearRegression不需要手动设置epochs和batch\_size,我们可以理解为一次全量训练。

\subsection{模型预测与参数查看}
查看模型参数及预测结果:
\begin{lstlisting}[language=Python]
print("截距:", model.intercept_)
print("系数:", model.coef_)
y_pred = model.predict(X)
\end{lstlisting}

\subsection{模型评估}
计算均方误差(MSE)与$R^2$分数:
\begin{lstlisting}[language=Python]
mse = mean_squared_error(y, y_pred)
r2 = r2_score(y, y_pred)
\end{lstlisting}

\subsection{结果可视化}
绘制真实函数、带噪声样本与预测线:
\begin{lstlisting}[language=Python]
plt.scatter(X, y, color='blue', label="带噪声样本")
plt.plot(X, y_true, color='green', label="真实函数")
plt.plot(X, y_pred, color='red', label="预测结果")
plt.legend()
plt.show()
\end{lstlisting}

\section{实验结果与分析}

\subsection{模型参数}
实验得到的模型参数如下:
\[
\hat{y} = -5.11494331x + 0.20754808
\]
可以看到参数接近真实值 $y = -5x + 0.1$,说明模型拟合相对良好。

\subsection{模型性能}
\begin{itemize}
    \item 均方误差 (MSE):0.2016461409917634
    \item $R^2$ 分数:0.9784809821540084
\end{itemize}
说明模型的预测性能非常优秀。

\subsection{可视化结果}
\begin{figure}[H]
\centering
\includegraphics[width=0.5\textwidth]{Figure_1.png}
\caption{线性回归结果可视化}
\end{figure}

可以看到模型拟合线(红色)与真实线(绿色)几乎重合,说明训练成功。

\section{实验结论}
本实验通过Scikit-learn的\texttt{LinearRegression}实现了线性回归模型的训练与评估,我掌握了:
\begin{itemize}
    \item 线性回归模型的基本原理;
    \item 模型训练、预测与评估流程;
    \item 均方误差与$R^2$指标的意义;
    \item 使用Matplotlib进行结果可视化。
\end{itemize}


\section{实验总结}
结合了上课老师所讲授的内容,通过本实验,我熟悉了 numpy 的数组操作、pandas 的数据处理以及 matplotlib 的可视化绘制。  

\newpage
\appendix
\section{完整代码}

\begin{lstlisting}[language=Python]
import numpy as np
import matplotlib.pyplot as plt
from sklearn.linear_model import LinearRegression
from sklearn.metrics import mean_squared_error, r2_score

plt.rcParams['font.sans-serif'] = ['SimHei']
plt.rcParams['axes.unicode_minus'] = False

np.random.seed(42)
X = 2 * np.random.rand(100, 1)
y_true = -5 * X + 0.1
noise = np.random.randn(100, 1) * 0.5
y = y_true + noise

model = LinearRegression()
model.fit(X, y)
print("模型截距:", model.intercept_)
print("模型系数:", model.coef_)

y_pred = model.predict(X)
mse = mean_squared_error(y, y_pred)
r2 = r2_score(y, y_pred)
print("均方误差:", mse)
print("R² 分数:", r2)

plt.figure(figsize=(4, 4))
plt.scatter(X, y, color='blue', label="样本数据", alpha=0.6)
plt.plot(X, y_true, color='green', label="y = -5x + 0.1")
plt.plot(X, y_pred, color='red', linewidth=2, label="模型预测线")
plt.title("线性回归拟合-武泽恺-10225101429")
plt.xlabel("x", fontsize=12)
plt.ylabel("y", rotation=0, fontsize=12)
plt.legend()
plt.show()
\end{lstlisting}

\end{document}