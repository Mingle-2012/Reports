\documentclass{article}
\usepackage{fancyhdr}
\usepackage{ctex}
\usepackage[a4paper, body={18cm,22cm}]{geometry}
\usepackage{amsmath,amssymb,amstext,wasysym,enumerate,graphicx}
\usepackage{float,abstract,booktabs,indentfirst,amsmath}
\usepackage{multirow}
\usepackage{enumitem}
\usepackage{xcolor}
\usepackage{tabularx}
\usepackage[most]{tcolorbox}
\usepackage{accsupp}
\usepackage[bottom]{footmisc}
\usepackage{subcaption}
\usepackage{logicproof}
% \usepackage[backend=biber,style=numeric]{biblatex}
\usepackage[xetex]{hyperref}
\usepackage{fontspec}
\usepackage{listingsutf8}
\usepackage{xcolor}
\usetikzlibrary{arrows.meta}
\newcommand\emptyaccsupp[1]{\BeginAccSupp{ActualText={}}#1\EndAccSupp{}}
\setlength{\parindent}{2em}
\setmonofont{Consolas}
\setCJKmonofont{黑体}
% \setmainfont{Times New Roman}
\hypersetup{CJKbookmarks=true,colorlinks=true,citecolor=blue,%
            linkcolor=blue,urlcolor=blue,bookmarksnumbered=true,%
            bookmarksopen=true,breaklinks=true}
\lstset{
    % language = C,
    inputencoding=utf8,
    extendedchars=false,
    showstringspaces=false,
    xleftmargin = 3em,xrightmargin = 3em, aboveskip = 1em,
	backgroundcolor = \color{white}, % 背景色
	basicstyle = \small\ttfamily, % 基本样式 + 小号字体
	rulesepcolor= \color{gray}, % 代码块边框颜色
	breaklines = true, % 代码过长则换行
	numbers = left, % 行号在左侧显示
	numberstyle=\emptyaccsupp,
    numbersep = 14pt, 
    keywordstyle=\color{purple}\bfseries, % 关键字颜色
    commentstyle =\color{red!50!green!50!blue!60}, % 注释颜色
    stringstyle = \color{red}, % 字符串颜色
    morekeywords={ASSERT, int64_t, uint32_t},
	% frame = shadowbox, % 用(带影子效果)方框框住代码块
	frame = single, % 用(带影子效果)方框框住代码块
	showspaces = false, % 不显示空格
	columns = fixed, % 字间距固定
  framesep=1em
} 
\lstset{
    sensitive=true,
    moreemph={ASSERT, NULL}, emphstyle=\color{red}\bfseries,
    moreemph=[2]{int64_t, uint32_t, tid_t, uint8_t, int16_t, uint16_t, int32_t, size_t}, emphstyle=[2]\color{purple}\bfseries,
    showspaces = false, % 不显示空格
    }

\raggedbottom

\title{\heiti\textbf{机器学习实践报告}}
\author{第三次实验 \\ 
武泽恺 10225101429
}
\date{2025年11月16日}


\begin{document}
\maketitle

\section{实验目的}
\begin{itemize}
    \item 理解逻辑回归的基本原理
    \item 学习利用sklearn训练逻辑回归模型
    \item 掌握模型训练、预测和评估流程
\end{itemize}

\section{实验环境}
Python 3.11.7,主要库包括:numpy、matplotlib、scikit-learn。

\section{实验内容与步骤}

\subsection{数据准备}

Iris数据集是模式识别最著名的数据集之一。数据集包含3类,每类50个实例,其中每个类都涉及一种鸢尾植物。 第一类与后两类可线性分离,后两类之间不能线性分离,所以本实验取前两类数据,做一个2分类数据集。我们使用 Iris 数据集前两类样本构建二分类任务,并取前两个特征进行可视化。

\subsection{数据读取与处理}
\begin{lstlisting}[language=Python]
iris = datasets.load_iris()
X = iris.data[:100, :2]
y = iris.target[:100]
X_train, X_test, y_train, y_test = train_test_split(
    X, y, test_size=0.2, random_state=11
)
\end{lstlisting}
取 40–60 部分数据观察样本特征与标签,如图~\ref{fig:iris_sample} 所示。

\begin{figure}[H]
    \centering
    \includegraphics[width=0.4\textwidth]{1.jpg}
    \caption{Iris 数据集样本展示(花萼长、花萼宽)}
    \label{fig:iris_sample}
\end{figure}

我们取样本的前两个属性进行2维可视化并输出,如图~\ref{fig:iris_2d} 所示。

随后使用 train\_test\_split 将数据按 8:2 划分训练集与测试集,并设置随机种子 11。

\begin{figure}[H]
    \centering
    \includegraphics[width=0.6\textwidth]{2.png}
    \caption{Iris 数据集前2维}
    \label{fig:iris_2d}
\end{figure}

\subsection{模型建立与训练}
使用 sklearn.linear\_model.LogisticRegression 构建模型:

\begin{lstlisting}[language=Python]
model = LogisticRegression(
    solver='lbfgs',
    C=1.0,
    max_iter=200
)
for _ in range(10):
    model.fit(X_train, y_train)
\end{lstlisting}

为了模拟 epoch=10,我们通过循环多次训练模型。

\subsection{模型预测与评估}
测试集精度计算公式如下:

\begin{equation}
    \text{Accuracy} = \frac{TP + TN}{TP + TN + FP + FN}
\end{equation}

可以通过 accuracy\_score 计算。

而F1值计算公式如下:

\begin{equation}
    F1 = 2 \cdot \frac{\text{Precision} \cdot \text{Recall}}{\text{Precision} + \text{Recall}}
\end{equation}

F1 值则可以通过 f1\_score 计算。

我们可以得到如图~\ref{fig:accuracy_f1} 所示的准确度与 F1 值。

\begin{figure}[H]
    \centering
    \includegraphics[width=0.3\textwidth]{6.png}
    \caption{准确度与 F1 值}
    \label{fig:accuracy_f1}
\end{figure}

我们通过网格计算模型预测值并使用 contourf 绘制决策区域,如图~\ref{fig:decision_boundary} 所示。

\begin{figure}[H]
    \centering
    \includegraphics[width=0.6\textwidth]{3.png}
    \caption{决策边界}
    \label{fig:decision_boundary}
\end{figure}

我们使用 TPR 与 FPR 计算并绘制 ROC 曲线,同时计算 AUC 值。AUC值的计算公式如下:

\begin{equation}
    \text{AUC} = \int_{0}^{1} \text{TPR} \, d(\text{FPR})
\end{equation}

如图~\ref{fig:roc_curve} 所示展示 ROC 曲线。
\begin{figure}[H]
    \centering
    \includegraphics[width=0.6\textwidth]{4.png}
    \caption{ROC 曲线}
    \label{fig:roc_curve}
\end{figure}

我们还计算了精确率与召回率并绘制精确率-召回率曲线展示模型在不同阈值下的性能。如图~\ref{fig:pr_curve} 所示。

\begin{figure}[H]
    \centering
    \includegraphics[width=0.6\textwidth]{5.png}
    \caption{PR 曲线}
    \label{fig:pr_curve}
\end{figure}

\section{实验结果}
模型测试集准确率为 1.0,F1 值也为 1.0,说明模型在测试集上表现完美。决策边界图显示两类样本被清晰分开。ROC 曲线和 PR 曲线均表现十分出色,说明模型拟合效果较好且特征具有较强区分能力。

\section{实验总结}
本实验中,我掌握了逻辑回归的基本原理、模型训练流程、模型性能评估方法以及可视化决策边界、PR 曲线、ROC 曲线的方法。Iris 前两类数据线性可分,因此逻辑回归模型表现十分优秀。


\newpage
\appendix
\section{完整代码}

\begin{lstlisting}[language=Python]
import numpy as np
import matplotlib.pyplot as plt
from sklearn import datasets
from sklearn.linear_model import LogisticRegression
from sklearn.model_selection import train_test_split
from sklearn.metrics import (
    accuracy_score, roc_curve, auc,
    precision_recall_curve, f1_score
)

plt.rcParams['font.sans-serif'] = ['SimHei']
plt.rcParams['axes.unicode_minus'] = False

iris = datasets.load_iris()
X = iris.data[:100, :2]
y = iris.target[:100]

print("40-60 数据:")
print(X[40:60], y[40:60])

plt.figure(figsize=(6, 5))
plt.scatter(X[:50, 0], X[:50, 1], c='red', label="类别 0")
plt.scatter(X[50:, 0], X[50:, 1], c='blue', label="类别 1")
plt.xlabel("花萼长")
plt.ylabel("花萼宽")
plt.title("Iris 数据集")
plt.legend()
plt.show()

X_train, X_test, y_train, y_test = train_test_split(
    X, y, test_size=0.2, random_state=11
)

model = LogisticRegression(
    solver='lbfgs',
    C=1.0,
    max_iter=200
)
for _ in range(10):
    model.fit(X_train, y_train)

y_pred = model.predict(X_test)
y_prob = model.predict_proba(X_test)[:, 1]

acc = accuracy_score(y_test, y_pred)
f1 = f1_score(y_test, y_pred)

print("测试集准确率:", acc)
print("F1 值:", f1)

plt.figure(figsize=(6, 5))

x_min, x_max = X[:, 0].min()-1, X[:, 0].max()+1
y_min, y_max = X[:, 1].min()-1, X[:, 1].max()+1
xx, yy = np.meshgrid(
    np.linspace(x_min, x_max, 300),
    np.linspace(y_min, y_max, 300)
)
Z = model.predict(np.c_[xx.ravel(), yy.ravel()])
Z = Z.reshape(xx.shape)
plt.contourf(xx, yy, Z, alpha=0.3, cmap=plt.cm.Paired)
plt.scatter(X[:50, 0], X[:50, 1], c='red')
plt.scatter(X[50:, 0], X[50:, 1], c='blue')
plt.title("决策边界")
plt.xlabel("花萼长")
plt.ylabel("花萼宽")
plt.show()

fpr, tpr, _ = roc_curve(y_test, y_prob)
roc_auc = auc(fpr, tpr)
plt.figure(figsize=(6, 5))
plt.plot(fpr, tpr, label="AUC = %.3f" % roc_auc)
plt.xlabel("假阳性率")
plt.ylabel("真阳性率")
plt.title("ROC 曲线")
plt.legend()
plt.show()

precision, recall, _ = precision_recall_curve(y_test, y_prob)
plt.figure(figsize=(6, 5))
plt.plot(recall, precision)
plt.xlabel("召回率")
plt.ylabel("精确率")
plt.title("精确率-召回率曲线")
plt.show()
\end{lstlisting}

\end{document}